\documentclass[12pt,a4paper]{article}
\usepackage[T1]{fontenc}
\usepackage[utf8]{inputenc}
\usepackage{authblk}
\usepackage[toc,page]{appendix}
\usepackage{amssymb, amsmath, latexsym}
\usepackage{footnote}
\usepackage{caption}
\usepackage{graphicx}
\usepackage{floatrow}
\floatsetup[table]{capposition=top}
\usepackage{caption,tabularx,booktabs}
\usepackage{mathtools}
\usepackage{lipsum}
\captionsetup{font=doublespacing}% Double-spaced float captions
\usepackage{setspace}   %Allows double spacing with the \doublespacing command

\doublespacing% Double-spaced document text
%\usepackage[titletoc]{appendix}
\usepackage{bbm}
\usepackage{subcaption}
\usepackage{bbding}
\usepackage[utf8]{inputenc}
\usepackage[english]{babel}
\usepackage{booktabs,caption}
\usepackage[flushleft]{threeparttable}
\usepackage{apacite}
\usepackage{natbib}
\usepackage{enumitem}
\usepackage{multirow}
\usepackage{tikz,tkz-tab}
\usepackage{bigstrut}
\usepackage{bigdelim}
\usepackage{tabvar}
\usepackage{setspace}
\usepackage[flushleft]{threeparttable}
\usepackage{booktabs}
\usepackage[gen]{eurosym}
\usepackage[colorlinks=true,
            linkcolor=red,
            urlcolor=blue,
            citecolor=blue]{hyperref}
\usepackage{bm}
\usepackage{ragged2e}
\usepackage{booktabs}
\usetikzlibrary{shapes,arrows,positioning}
\usepackage{lscape}
\usepackage{color}
\usepackage{alltt}
\newcommand*{\threeemdash}{\rule[0.5ex]{3em}{0.55pt}}
\usepackage{xparse}
\usepackage{amsmath}
\newcommand{\norm}[1]{\left\lVert#1\right\rVert}
\usepackage[short]{optidef}
\newcommand*{\xdash}[1][6em]{\rule[0.7ex]{#1}{0.75pt}}
%\bibliographystyle{abbrvnat}
%\setcitestyle{authoryear,open={((},close={))}}
% Page length commands go here in the preamble
\setlength{\oddsidemargin}{-0.25in} % Left margin of 1 in + 0 in = 1 in
\setlength{\textwidth}{7in}   % Right margin of 8.5 in - 1 in - 6.5 in = 1 in
\setlength{\topmargin}{-.75in}  % Top margin of 2 in -0.75 in = 1 in
\setlength{\textheight}{9.2in}  % Lower margin of 11 in - 9 in - 1 in = 1 in

\newtheorem{theorem}{Theorem}
\newtheorem{definition}{Definition}
\newtheorem{lemma}{Lemma}
\usepackage{booktabs,array,lmodern}
\usepackage{hyperref}
%\addbibresource{reference.bib}

\renewcommand{\baselinestretch}{1.5} % 1.5 denotes double spacing. Changing it will change the spacing
\renewcommand\Authands{ and }
\setlength{\parindent}{0in}


\newtheorem{hyp}{Hypothesis}
\newtheorem{assumption}{Assumption}
\newtheorem{subhyp}{Hypothesis}[hyp]
\renewcommand{\thesubhyp}{\thehyp\alph{subhyp}}

\newcommand{\red}[1]{{\color{red} #1}}
\newcommand{\blue}[1]{{\color{blue} #1}}

\begin{document}

\begin{titlepage}
\title{Macroeconomics 2-Assignment 2 }
\end{titlepage}
%\date{\today}
\author{Wilfried Youmbi\\} 
\maketitle



\section{Define a competitive equilibrium for this economy}
A competitive equilibrium for this economy comprises of price
$\{p_{t}, w_{t}, r_{t}\}^{\infty}_{t=0}$, allocations for the firm $\{k^{d}_{t}, l^{d}_{t}, y_{t}\}^{\infty}_{t=0}$ and the household $\{c_{t}, i_{t}, x_{t+1}, k^{s}_{t}, l^{s}_{t}\}^{\infty}_{t=0}$ such that:

\begin{itemize}
\item Given prices $\{p_{t}, w_{t}, r_{t}\}^{\infty}_{t=0}$, the allocation of the representative firm $\{k^{d}_{t}, l^{d}_{t}, y_{t}\}^{\infty}_{t=0}$ solves:\\



    \begin{maxi}
	  {\{k^{d}_{t}, l^{d}_{t}, y_{t}\}_{t=0}^\infty}{\sum_{t=0}^{\infty} p_{t}(y_{t}-r_{t}k^{d}_{t}-w_{t}l^{d}_{t})}{}{}	  
 \addConstraint {y_{t}}{=z(k^{d}_{t})^\alpha(l^{d}_{t})^{1-\alpha},\quad} \forall t \geq 0
   \addConstraint{y_{t}, k^{d}_{t}, l^{d}_{t} \geq 0}~\forall t \geq 0
    \end{maxi}
  
  \item Given prices $\{p_{t}, w_{t}, r_{t}\}^{\infty}_{t=0}$, the allocation of the representative household $\{c_{t}, i_{t}, x_{t+1}, k^{s}_{t}, l^{s}_{t}\}^{\infty}_{t=0}$ solves:
  
   \begin{maxi}
	  {\{c_{t}, i_{t}, x_{t+1}, k^{s}_{t}, l^{s}_{t}\}_{t=0}^\infty}{\sum_{t=0}^{\infty} \beta_{t}(\frac{(c_{t})^{1-\sigma}}{1-\sigma}-\chi \frac{(l^s_{t})^{1+\eta}}{1+\eta})}{}{}	  
 \addConstraint {\sum_{t=0}^{\infty}p_{t}(c_{t}+i_{t}) }{\leq \sum_{t=0}^{\infty} p_{t}(r_{t}k^{s}_{t} + w_{t}l^{s}_{t}) + \pi ,\quad} \forall t \geq 0
 \addConstraint {x_{t+1}}{=(1-\delta)x_{t}+i_{t},\quad} \forall t \geq 0
 \addConstraint{0\leq l^{s}_{t}\leq 1, ~0 \leq k^{s}_{t}\leq x_{t} \quad }~\forall t \geq 0
 \addConstraint{c_{t}, x_{t+1} \geq 0}~\forall t \geq 0
 \addConstraint{k_{0} ~~~given}
    \end{maxi}
  
\item Markets clearing conditions
\begin{itemize}
\item    $y_{t} = c_{t} + i_{t}$ (Goods Market)
\item  $ l^{d}_{t} = l^{s}_{t} = l_{t}$ (Labour Market)
\item $k^{d}_{t} = k^{s}_{t} =k_{t}$ (Capital Services Market)
\end{itemize}
\end{itemize}

\section{Find the steady state value for \{c, l, k, y, r, w\}}


    \begin{maxi}
	  {\{k^{d}_{t}, l^{d}_{t}, y_{t}\}_{t=0}^\infty}{\sum_{t=0}^{\infty} p_{t}(y_{t}-r_{t}k^{d}_{t}-w_{t}l^{d}_{t})}{}{}	  
 \addConstraint {y_{t}}{=z(k^{d}_{t})^\alpha(l^{d}_{t})^{1-\alpha},\quad} \forall t \geq 0
   \addConstraint{y_{t}, k^{d}_{t}, l^{d}_{t} \geq 0}~\forall t \geq 0
    \end{maxi}
  
The first order conditions of the firm's problem with respect to $k_{t}$ and $l_{t}$ are the following:
    \begin{equation}
    \begin{split}
    r_{t} &= z\alpha (k_{t})^{\alpha-1}(l_{t})^{1-\alpha}\\
    w_{t} &= z(1-\alpha)(k_t)^{\alpha} (l_t)^{-\alpha}\\
    \pi &= \sum_{t=0}^{\infty} p_{t}(z(k_{t})^\alpha(l_{t})^{1-\alpha}-r_{t}k_{t}-w_{t}l_{t})=0
    \end{split}
    \end{equation}
  
   \begin{maxi}
	  {\{c_{t}, i_{t}, x_{t+1}, k^{s}_{t}, l^{s}_{t}\}_{t=0}^\infty}{\sum_{t=0}^{\infty} \beta_{t}(\frac{(c_{t})^{1-\sigma}}{1-\sigma}-\chi \frac{(l^s_{t})^{1+\eta}}{1+\eta})}{}{}	  
 \addConstraint {\sum_{t=0}^{\infty}p_{t}(c_{t}+i_{t}) }{\leq \sum_{t=0}^{\infty} p_{t}(r_{t}k^{s}_{t} + w_{t}l^{s}_{t}) + \pi ,\quad} \forall t \geq 0
 \addConstraint {x_{t+1}}{=(1-\delta)x_{t}+i_{t},\quad} \forall t \geq 0
 \addConstraint{0\leq l^{s}_{t}\leq 1, ~0 \leq k^{s}_{t}\leq x_{t} \quad }~\forall t \geq 0
 \addConstraint{c_{t}, x_{t+1} \geq 0}~\forall t \geq 0
 \addConstraint{k_{0} ~~~given}
    \end{maxi}
    
 \begin{equation}
    \begin{split}
    \mathcal{L}(\{c_t,k_t,l_t\}_{t=0}^{\infty},\lambda_t)=&\sum_{t=0}^{\infty} \beta ^t(\frac{c_t^{1-\sigma}}{1-\sigma}-\chi \frac{l_t^{1+\eta}}{1+\eta})\\
    &+\lambda_t(\sum_{t=0}^{\infty}p_t(zk_t^{\alpha}l_t^{1-\alpha}-c_t-k_{t+1}+(1-\delta)k_t)\\
    \end{split}
    \end{equation}
    
  The first order conditions are the following:
  
  \begin{equation}
    \begin{split} 
    \frac{\partial \mathcal{L}}{\partial c_t} &=\beta^tc_t^{-\sigma}-\lambda_t p_t=0\\
    \frac{\partial \mathcal{L}}{\partial l_t} &=-\beta^t\chi l_t^{\eta}+\lambda_t p_tz(1-\alpha) k_t^{\alpha}l_t^{-\alpha}=0\\
    \frac{\partial \mathcal{L}}{\partial k_t}&=\lambda_t p_t(z\alpha k_t^{\alpha -1}l_t^{1-\alpha}+1-\delta)-\lambda_{t-1}p_{t-1}=0\\
    \frac{\partial \mathcal{L}}{\partial \lambda_t} &= zk_t^{\alpha}l_t^{1-\alpha}-k_{t+1}+(1-\delta)k_t - c_t = 0
    \end{split}
    \end{equation}
    
  In steady state, $c_t =c, k_t =k, l_t =l$. Normalizing  $p_0=1$, we get:
 \begin{equation}
    \begin{split}
    \lambda_t &= c\\
    p_t &= \beta^t\\
    c^{\sigma} l^{\eta}&= z(1-\alpha)k^{\alpha}l^{-\alpha}/\chi\\
    z\alpha k^{\alpha -1}l^{1-\alpha}&= 1/\beta -1 +\delta\\
    c&=zk^{\alpha}l^{1-\alpha}-\delta k
    \end{split}
    \end{equation}
 
Next, we will express the steady state variables in term of $U=\frac{k}{l}$ and $V=\frac{c}{l}$.

  \begin{equation}
    \begin{split}
    \: U &= \frac{k}{l}=\left(\frac{z\alpha \beta}{1-\beta +\beta \delta}\right)^{\frac{1}{1-\alpha}} \\
   \: V &=\frac{c}{l}=z(\frac{k}{l})^{\alpha}-\delta \frac{k}{l}=zU^{\alpha}-\delta U \\
    c^{\sigma} l^{\eta} &= (lV)^{\sigma}l^{\eta}\\
    &= z(1-\alpha)(\frac{k}{l})^{\alpha}/\chi\\
    &\Rightarrow l^{\sigma+\eta} =\frac{z(1-\alpha)U^{\alpha}}{\chi V^{\sigma}}\\ &\Rightarrow l=\left(\frac{z(1-\alpha)U^{\alpha}}{\chi V^{\sigma}}\right)^{\frac{1}{\sigma+\eta}} \quad (*)
    \end{split}
    \end{equation}
Having $l$ in steady state, we can obtain other steady state variables:
  \begin{equation}
    \begin{split}
    k&=Ul\quad (*)\\
    c&=Vl\quad (*)\\
    y&=zk^{\alpha}l^{1-\alpha}=zU^{\alpha}l\quad (*)\\
    r&=\alpha zk^{\alpha-1}l^{1-\alpha}=\alpha zU^{\alpha -1}\quad (*)\\
    w&=(1-\alpha)zk^{\alpha}l^{-\alpha}=(1-\alpha)zU^{\alpha} \quad (*)
    \end{split}
    \end{equation}
    
     \section{Pose the planner’s dynamic programming problem. Write down the appropriate
Bellman equation.}

The social planner's problem is as follows:
\begin{equation}
\begin{split}
\max\limits_{\{k_{t}, c_{t}, l_{t}\}^{\infty}_{t=0}} \sum_{t=0}^{\infty} \beta^t(\frac{(c_{t})^{1-\sigma}}{1-\sigma}-\chi \frac{(l_{t})^{1+\eta}}{1+\eta}) \\
    s.t. \quad z(k_t)^{\alpha}(l_t)^{1-\alpha} &= c_t+k_{t+1}-(1-\delta)k_t \quad \forall t\geq0\\
    c_t &\geq 0, k_t \geq 0, 0 \leq l_t \leq 1 \quad \forall t\geq0\\
     \quad {k}_{0}\:given
\end{split}
\end{equation}
The associated Bellman equation is defined as follows:
\begin{equation}
\begin{split}
v(k_0) = \max\limits_{\begin{smallmatrix}0\leq k' \leq z(k)^{\alpha}(l)^{1-\alpha}+ (1-\delta)k\\ 0\leq l \leq 1\end{smallmatrix}} \{\frac{(z(k)^{\alpha}(l)^{1-\alpha}+ (1-\delta)k-k')^{1-\sigma}}{1-\sigma}-\chi \frac{(l)^{1+\eta}}{1+\eta}+\beta v(k')\}
\end{split}
\end{equation}

\section{Find $\chi$ such that $l_{ss}$ = 0.4}

From $l_{ss}^{\sigma+\eta} =\frac{z(1-\alpha)U^{\alpha}}{\chi V^{\sigma}}$, we can derive $\chi $ as: $\chi = \frac{z(1-\alpha)U^{\alpha}}{l_{ss}^{\sigma+\eta} V^{\sigma}}$ \\


Plug in $\alpha=1/3,z=1,\delta=2,\eta=1,l_{ss}=0.4$, $\beta=0.9$ we obtain
\begin{equation}
\chi = 38.81
\end{equation}

\section{Solve the planner’ problem numerically using value function iteration. You must do it using:}
\subsection{Plain VFI}

\begin{figure}[!htbp]
\centering
  \includegraphics[scale=0.45]{ASS2_5a_policy.png}
  \caption{Approximated Value Function}
  \label{fig:boat1}
\end{figure}
\pagebreak  

\begin{figure}[!htbp]
\centering
  \includegraphics[scale=0.45]{ASS2_5a_value.png}
  \caption{Approximated Policy Function}
  \label{fig:boat1}
\end{figure}

\begin{figure}[!htbp]
\centering
  \includegraphics[scale=0.45]{ASS2_5a_Euler.png}
  \caption{Approximated Error($\%$)}
  \label{fig:boat1}
\end{figure}

\subsection{Modified Howard’s Policy Iteration (you must choose the number of policy iterations)}

\pagebreak
\begin{figure}[!htbp]
\centering
  \includegraphics[scale=0.45]{ASS2_5b_policy.png}
  \caption{Approximated Value Function}
  \label{fig:boat1}
\end{figure}

\begin{figure}[!htbp]
\centering
  \includegraphics[scale=0.45]{ASS2_5b_value.png}
  \caption{Approximated Policy Function}
  \label{fig:boat1}
\end{figure}
\pagebreak

\pagebreak
\begin{figure}[!htbp]
\centering
  \includegraphics[scale=0.45]{ASS2_5b_Euler.png}
  \caption{Approximated Error($\%$)}
  \label{fig:boat1}
\end{figure}

\subsection{MacQueen-Porteus Bounds}
\begin{figure}[!htbp]
\centering
  \includegraphics[scale=0.45]{ASS2_5c_policy.png}
  \caption{Approximated Value Function}
  \label{fig:boat1}
\end{figure}

\begin{figure}[!htbp]
\centering
  \includegraphics[scale=0.45]{ASS2_5c_value.png}
  \caption{Approximated Policy Function}
  \label{fig:boat1}
\end{figure}

\begin{figure}[!htbp]
\centering
  \includegraphics[scale=0.45]{ASS2_5c_Euler.png}
  \caption{Approximated Error($\%$)}
  \label{fig:boat1}
\end{figure}
\pagebreak

\section{Use the solution to the planner’s problem to obtain the steady state value of $\{c, k, r,l,w, y\}$}
\subsection{Capital decreases to 80$\%$ of its steady state value}
\begin{figure}[!htbp]
\centering
  \includegraphics[scale=0.5]{ASS2_6a_k.png}
  \caption{SS capital decreases to 80$\%$ }
  \label{fig:boat1}
\end{figure}

\subsection{Productivity increases permanently by 5$\%$}
\begin{figure}[!htbp]
\centering
  \includegraphics[scale=0.5]{ASS2_6a_z.png}
  \caption{Productivity increases permanently by 5$\%$ }
  \label{fig:boat1}
\end{figure}

\section{Prove that the mapping used in Howard’s policy iteration algorithm is a contraction.}
I will prove this using an alternative version of Blackwell’s theorem\\
\begin{equation}
\begin{split}
Tv(k) = \max\limits_{\begin{smallmatrix}0\leq k' \leq z(k)^{\alpha}(l)^{1-\alpha}+ (1-\delta)k\\ 0\leq l \leq 1\end{smallmatrix}} \{\frac{(z(k)^{\alpha}(l)^{1-\alpha}+ (1-\delta)k-k')^{1-\sigma}}{1-\sigma}-\chi \frac{(l)^{1+\eta}}{1+\eta}+\beta v(k')\}
\end{split}
\end{equation}

Let's pose 
\begin{equation*}
U(f(k)-k', l)=\frac{(f(k)-k')^{1-\sigma}}{1-\sigma}-\chi \frac{l^{1+\eta}}{1+\eta}
\end{equation*}

Let's define $(B[0,\infty),d)$ as our metric space. $B$ being the space of bounded functions on $[0, \infty)$ with $d$ being the sup-norm. I am going to verify that all the hypotheses for Blackwell’s theorem are satisfied.

\begin{itemize}
\item T maps $(B[0,\infty),d)$ into itself. 

In fact, if we take $v$ to be bounded, since we
assumed that U is bounded, so is $Tv$.
\item Monotonicity

Given $v$, $w$  $\in  B[0,\infty)$, $\beta \in (0,1)$, and $k \geq 0$ such that $v(k) \leq w(k)$, if $\overline{g}(k)$ denotes the Howard's optimal policy function, hence,
\begin{equation}
\begin{split}
Tv(k) &=  U(\overline{g}(k))+\beta v(\overline{g}(k))\\
&\leq U(\overline{g}(k))+\beta w(\overline{g}(k))\\
&= Tw(k)
\end{split}
\end{equation}
\item Discounting

Given $v \in B[0,\infty)$ , $k \geq 0$, $a \geq 0$, and $\beta \in (0,1)$
\begin{equation}
\begin{split}
T(v+a)(k) &=  U(\overline{g}(k))+\beta (v(\overline{g}(k))+a)\\
&= U(\overline{g}(k))+\beta v(\overline{g}(k))+\beta a\\
&= Tv(k) + \beta a
\end{split}
\end{equation}
\end{itemize}

The mapping used in Howard's fixed policy function satisfies all the hypotheses for Blackwell’s theorem, therefore it's a contraction mapping.

\end{document}
