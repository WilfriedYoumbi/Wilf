\documentclass[12pt,a4paper]{article}
\usepackage[T1]{fontenc}
\usepackage[utf8]{inputenc}
\usepackage{authblk}
\usepackage[toc,page]{appendix}
\usepackage{amssymb, amsmath, latexsym}
\usepackage{footnote}
\usepackage{caption}
\usepackage{graphicx}
\usepackage{floatrow}
\floatsetup[table]{capposition=top}
\usepackage{caption,tabularx,booktabs}
\usepackage{mathtools}
\usepackage{lipsum}
\captionsetup{font=doublespacing}% Double-spaced float captions
\usepackage{setspace}   %Allows double spacing with the \doublespacing command

\doublespacing% Double-spaced document text
%\usepackage[titletoc]{appendix}
\usepackage{bbm}
\usepackage{subcaption}
\usepackage{bbding}
\usepackage[utf8]{inputenc}
\usepackage[english]{babel}
\usepackage{booktabs,caption}
\usepackage[flushleft]{threeparttable}
\usepackage{apacite}
\usepackage{natbib}
\usepackage{enumitem}
\usepackage{multirow}
\usepackage{tikz,tkz-tab}
\usepackage{bigstrut}
\usepackage{bigdelim}
\usepackage{tabvar}
\usepackage{setspace}
\usepackage[flushleft]{threeparttable}
\usepackage{booktabs}
\usepackage[gen]{eurosym}
\usepackage[colorlinks=true,
            linkcolor=red,
            urlcolor=blue,
            citecolor=blue]{hyperref}
\usepackage{bm}
\usepackage{ragged2e}
\usepackage{booktabs}
\usetikzlibrary{shapes,arrows,positioning}
\usepackage{lscape}
\usepackage{color}
\usepackage{alltt}
\newcommand*{\threeemdash}{\rule[0.5ex]{3em}{0.55pt}}
\usepackage{xparse}
\usepackage{amsmath}
\newcommand{\norm}[1]{\left\lVert#1\right\rVert}
\usepackage[short]{optidef}
\newcommand*{\xdash}[1][6em]{\rule[0.7ex]{#1}{0.75pt}}
%\bibliographystyle{abbrvnat}
%\setcitestyle{authoryear,open={((},close={))}}
% Page length commands go here in the preamble
\setlength{\oddsidemargin}{-0.25in} % Left margin of 1 in + 0 in = 1 in
\setlength{\textwidth}{7in}   % Right margin of 8.5 in - 1 in - 6.5 in = 1 in
\setlength{\topmargin}{-.75in}  % Top margin of 2 in -0.75 in = 1 in
\setlength{\textheight}{9.2in}  % Lower margin of 11 in - 9 in - 1 in = 1 in

\newtheorem{theorem}{Theorem}
\newtheorem{definition}{Definition}
\newtheorem{lemma}{Lemma}
\usepackage{booktabs,array,lmodern}
\usepackage{hyperref}
%\addbibresource{reference.bib}

\renewcommand{\baselinestretch}{1.5} % 1.5 denotes double spacing. Changing it will change the spacing
\renewcommand\Authands{ and }
\setlength{\parindent}{0in}


\newtheorem{hyp}{Hypothesis}
\newtheorem{assumption}{Assumption}
\newtheorem{subhyp}{Hypothesis}[hyp]
\renewcommand{\thesubhyp}{\thehyp\alph{subhyp}}

\newcommand{\red}[1]{{\color{red} #1}}
\newcommand{\blue}[1]{{\color{blue} #1}}

\begin{document}

\begin{titlepage}
\title{Macroeconomics 2-Assignment 1 }
\end{titlepage}
%\date{\today}
\author{Wilfried Youmbi\\} 
\maketitle

\section{Define a competitive equilibrium for this economy}

A competitive equilibrium for this economy comprise of price $\{p_{t}, w_{t}, r_{t}\}_{t=0}^\infty$, and allocations for the firm $\{k^{d}_{t}, n^{d}_{t}, y_{t}\}_{t=0}^\infty$, and the household $\{c_{t}, i_{t}, x_{t+1}, k_{t}^{s}, n_{t}^{s}\}_{t=0}^\infty$ such that:

\begin{itemize}
\item Given $\{p_{t}, w_{t}, r_{t}\}_{t=0}^\infty$, the allocation of the representative firm $\{k^{d}_{t}, n^{d}_{t}, y_{t}\}_{t=0}^\infty$ solves:

    \begin{maxi}
	  {\{y_{t}, k_{t}, n_{t}\}_{t=0}^\infty}{\Sigma_{t=0}^{\infty} p_{t}(y_{t}-r_{t}k_{t}-n_{t}w_{t})}{}{}	  
 \addConstraint {y_{t}}{=F(k_{t}, n_{t}),\quad}~for~ all~t \geq 0
   \addConstraint{y_{t}, k_{t}, n_{t} \geq 0}
    \end{maxi}
    
 \item Given $\{p_{t}, w_{t}, r_{t}\}_{t=0}^\infty$, the allocation of the representative household $\{c_{t}, i_{t}, x_{t+1}, k_{t}^{s}, n_{t}^{s}\}_{t=0}^\infty$ solves:
 
    \begin{maxi}
	  {\{c_{t}, i_{t}, x_{t+1}, k_{t}^{s}, n_{t}^{s}\}_{t=0}^\infty}{\Sigma_{t=0}^{\infty} \beta^{t}U(c_{t})}{}{}	  
 \addConstraint{\Sigma_{t=0}^{\infty} p_{t}(c_{t}+i_{t})}{=\Sigma_{t=0}^{\infty} p_{t}(r_{t}k_{t}+n_{t}w_{t}))+\pi,\quad}~for~ all~t \geq 0
  \addConstraint {x_{t+1}}{=(1-\delta)x_{t}+i_{t},\quad}~for~ all~t \geq 0
   \addConstraint{0\leq n_{t} \leq 1,~ 0\leq k_{t} \leq x_{t},}
    \addConstraint{c_{t}, x_{t+1} \geq 0}~for~ all~t \geq 0
     \addConstraint{x_{0}~~given}  
    \end{maxi}
    
\item Markets clearing
\begin{itemize}
\item $y_{t}=c_{t}+i_{t}$ (Goods Markets)
\item  $n^{d}_{t}=n^{s}_{t}$ (Labour Markets)
\item $k^{d}_{t}=k^{s}_{t}$ (Capital Services Markets)
\end{itemize}
\end{itemize}


\section{Define the social planner’s problem for this economy.}
The social planner's problem for this economy is defined as follows:


    \begin{maxi}
	  {\{c_{t}, k_{t}, n_{t}\}_{t=0}^\infty }{\Sigma_{t=0}^{\infty} \beta^{t}U(c_{t})}{}{}	  
 \addConstraint {F(k_{t}, n_{t})}{=c_{t}+k_{t}-(1-\delta)k_{t},\quad}~for~ all~t \geq 0
   \addConstraint{y_{t}, k_{t}, n_{t} \geq 0}
    \addConstraint{c_{t} \geq 0,~ k_{t} \geq 0, ~0\leq n_{t} \leq 1,}
       \addConstraint{k_{0} \leq \bar{k_{0} }}
    \end{maxi}
The value function is the total lifetime utility of the representative household if the social planner chooses  $\{y_{t}, k_{t}, n_{t}\}_{t=0}^\infty$   optimally and the initial capital stock in the economy is $\bar{k_{0} }$.
  \section{Show that the equilibrium allocation of consumption, capital, and labor coincides with those of the planner’s.}  
 By the first welfare theorem, equilibrium allocation of consumption, capital, and labor coincides with those of the planner’s. In fact, household's preference satisfies local non-satiation condition and the first welfare theorem states that any competitive equilibrium leads to a Pareto efficient allocation of resources under local non-satiation of household preference. 
 
 By local non-satiation we mean that, for any consumption allocation $x$ and any $\epsilon >0$, there's some allocation $x'$ such that $\norm{x-x'} \leq \epsilon $ and $U(x')>U(x)$.
 
 \section{Pose the planner’s dynamic programming problem. Write down the appropriate Bellman equation.}Let's assume U to be continuously differentiable, strictly increasing, strictly concave and bounded. In addition, let's assume that U satisfies Inada conditions. The discount factor $\beta$ belongs to $(0, 1)$.
 
 Furthermore, let's assume that F is continuously differentiable and homogenous of degree 1; strictly increasing and strictly concave.  $F(0,n)=F(k,0)=0$ for all $k, n >0$. Let's assume also that F satisfies Inada conditions.

From these assumptions, it turns out that $n_t = 1$ and $k_{0} =\bar{k_{0}}$ for all t since households do not value leisure in their utility function and production function is strictly increasing in capital. 

To simplify notation we define $f(k) = F(k, 1) + (1-\delta)k$ for all k. Exploiting the implications of the assumptions, and substituting for $c_{t}=f(k_{t})-k_{t+1}$, ~we can rewrite the previous social planner's problem as:

   \begin{maxi}
	  {\{k_{t+1}\}_{t=0}^\infty}{\Sigma_{t=0}^{\infty} \beta^{t}U(f(k_{t})-k_{t+1})}{}{}	  
   \addConstraint{0\leq k_{t+1} \leq f(k_{t})}
       \addConstraint{k_{0}=\bar{k_{0}}>0}
    \end{maxi}

The planner faces between letting the consumer eat today versus investing in the capital stock.

The associated Bellman equation is defined as:

$V(k)=Max \{U(f(k)-k')+\beta V(k')\}$,~~~ with $0 \leq k' \leq f(k)$.

\section{Solve the planner’s dynamic programming
problem (find the value and policy functions).}

Pose $u (c) = \ln c$ and $f(k,l)=zk^{\alpha}l^{1-\alpha}$\\
The Bellman becomes: $V(k)=Max \{\ln(f(k)-k')+\beta \ln(k')\}$~$0 \leq k' \leq zk^{\alpha}$.

I am going to guess a particular functional form of a solution and then verify that the solution has in fact this form.

Let's pose V(k)=A+B lnk

Let's find the value of $k'$ that maximises $ln(zk^{\alpha}-k')+\beta(A+Bln(k'))$. 
From the first order condition we find that: $\frac{1}{zk^{\alpha}-k'}=\frac{\beta B}{k'}$,~~~Hence, $k'=\frac{zk^\alpha\beta B}{1+\beta B}$.

$A+B lnk=ln(zk^\alpha-k')+\beta (A+B \ln(k'))$

Plugging the expression of $k'$ in the above equality and proceed by identification, we find $B=\frac{\alpha}{1-\alpha\beta}$.

Plugging back the expression of B in the above equality, we obtain.

$A(1-\beta)=\frac{\alpha\beta}{1-\alpha\beta}ln(\alpha\beta)+ln(1-\alpha\beta)+\alpha \frac{ln z}{1-\alpha\beta}$~~~Hence, $A=\frac{1}{1-\beta}\big(\frac{\alpha\beta}{1-\alpha\beta} \ln(\alpha\beta)+\ln(1-\alpha\beta)+\alpha \frac{ln z}{1-\alpha\beta} \big)$ \\

Therefore, the value function is $V^*(k)=A+B \ln(k)$~~with A, B
as determined above. The associated optimal policy function is $k'=\frac{zk^\alpha\beta B}{1+\beta B}=g(k)$. Plugging in the expression of B we obtain $g(k)=z\alpha \beta k^\alpha $

\section{Use the solution to the planner’s problem to obtain the steady state value of \{c, k, r, w, y\}}

Solving the Social Planner's problem, we obtain the following first order condition:

$\frac{-\beta^{t}}{zk^{\alpha}_{t}-k_{t+1}}-\frac{\beta^{t+1} \alpha zk^{\alpha}_{t+1}}{zk^{\alpha}_{t+1}-k_{t+2}}=0$  and so the Euler equation is $c_{t+1}=\alpha \beta z k_{t+1}^{\alpha-1} c_{t}$.

At the steady state, $c_{t+1}=c_{t}=c^*$ and $k_{t+1}=k_{t}=k^*$. ~Hence, $k^*=(\alpha \beta z)^{\frac{1}{1-\alpha}}$

Using the resource constraint at the steady state, we obtain $c^*=f(k^*)-k^*=zk^{*^{\alpha}}-k^*$ 

So~~$c^*=z(\alpha \beta z)^{\frac{\alpha}{1-\alpha}}-(\alpha \beta z)^{\frac{1}{1-\alpha}}$,~~$r*=f'(k^*)=\alpha zk^{*^{\alpha-1}}$,~~and

$w*=z(1-\alpha)k^{*^{\alpha}}l^{-\alpha}$; ~~$y*=r^*k^*-w^{*}l$

\section{ Assume that  $\alpha=\frac{1}{3}$, $z = 1$. Use the solution to the planner’s problem to obtain the path of \{c, k, r, w, y\} starting from the steady state after the following changes}

$k^*=(\frac{1}{3}\beta )^{\frac{3}{2}}$;~~\\
$c^*=(\frac{1}{3}\beta )^{\frac{1}{2}}-(\frac{1}{3}\beta )^{\frac{3}{2}}=(\frac{1}{3}\beta )^{\frac{1}{2}}\big(1-\frac{1}{3}\beta\big)$;\\ ~~$r^*=f'(k^*)=\frac{1}{3}\beta k^{*^{\frac{-2}{3}}}=\beta^{-1}$;  \\
$w^*=\frac{2}{3}k^{*^{\frac{1}{3}}}$\\
$y^*=r^*k^*-w^{*}$

\begin{enumerate}
\item \textbf{Capital decreases to $80\%$ of its steady state value}

\begin{figure}[htbp]
\centerline{\includegraphics[scale=0.7]{HW17a}}
\label{fig}
\end{figure}

\item \textbf{Productivity increases permanently by $5\%$}

\begin{figure}[htbp]
\centerline{\includegraphics[scale=0.7]{HW17a}}
\label{fig}
\end{figure}

\end{enumerate}



\end{document}